\documentclass{article}

\usepackage[utf8]{inputenc}

% Packages
\usepackage{amsmath,amssymb}
\usepackage{bm}% boldmath
\usepackage{listings} % Code block (source code) \begin{lstlisting} 
\usepackage{natbib}
\usepackage{graphicx}
\usepackage{lmodern}
\usepackage[usenames,dvipsnames,svgnames,table]{xcolor}
\usepackage[textwidth=16cm,textheight=23cm]{geometry}

%\usepackage{inconsolata} % New monospace font

% URL
\usepackage{url}
\usepackage[colorlinks=true, a4paper=true, pdfstartview=FitV, linkcolor=blue, citecolor=blue, urlcolor=blue]{hyperref}

% Figures
\usepackage[font=small, labelfont=bf]{caption}
\usepackage{subfig} % Subfigures. Uses \subfloat[captions text]{figure}

% Tables
\usepackage{booktabs}   % Allows the use of \toprule, \midrule and \bottomrule in tables for horizontal lines
\newcommand{\ra}[1]{\renewcommand{\arraystretch}{#1}} % spaces in tables

% Itemize
\usepackage{enumitem}

% Commands
%\newcommand{\code}[1]{\texttt{#1}} % \code{inline code}
\newcommand{\code}[1]{{\small\ttfamily #1}} % \code{inline code}
\newcommand{\expval}[1]{\langle #1 \rangle} %
\renewcommand{\theequation}{\arabic{section}.\arabic{equation}} % Book format equation
\renewcommand{\thefigure}{\arabic{section}.\arabic{figure}} % Book format figure
\renewcommand{\vec}[1]{{\bf #1}} % Lars likes this better than arrow

% Set page attribution
\setlength{\parindent}{0pt}


% PSTRICKS
\usepackage{pstricks,pst-node,pst-tree} % includes graph additions
\usepackage{pst-pdf} % Compiles the pictures
\usepackage{pst-node}
\usepackage{pst-plot}
\usepackage{pst-3dplot}
%\usepackage{pstricks-add,babel}




\lstset{
language=Python,                        % Code langugage
commentstyle=\color{gray},              % Comments font
basicstyle=\small\ttfamily,             % Code font, Examples: \footnotesize, \ttfamily
keywordstyle=\bfseries\color{blue},
stringstyle=\color{orange},
numbers=left,                           % Line nums position
numberstyle=\tiny,                      % Line-numbers fonts
stepnumber=1,                           % Step between two line-numbers
numbersep=5pt,                          % How far are line-numbers from code
frame=none,                             % A frame around the code
tabsize=4,                              % Default tab size
captionpos=b,                           % Caption-position = bottom
breaklines=true,                        % Automatic line breaking?
breakatwhitespace=false,                % Automatic breaks only at whitespace?
showspaces=false,                       % Dont make spaces visible
showstringspaces=false,                 % Dont make spaces visible in strings
showtabs=false,                         % Dont make tabls visible
belowskip=8pt,
morekeywords={range, xrange},
% backgroundcolor=\color{yellow}
% emph={[2]root,base}
% morekeywords={one,two,three,four,five,six,seven,eight,
}


%commentstyle=\color{gray},              % Comments font
%basicstyle=\small,                      % Code font, Examples: \footnotesize, \ttfamily



%basicstyle=\footnotesize\ttfamily,
%keywordstyle=\bfseries\color{green!40!black},
%commentstyle=\itshape\color{purple!40!black},
%identifierstyle=\color{blue},
%stringstyle=\color{orange},







% ***************************************************
% HEADER INFORMATION

\title{Exercise 5}
\author{Molecular Statistics, Week 5}
\date{2014}

% ***************************************************

\begin{document}


% ***************************************************
% BEGIN DOCUMENT
% ***************************************************

\maketitle

\section{Introduction}

Often it will be necessary to data-mine, manipulate and visualize data obtained manually from experiments or from other software.
Python is great for this and the goals of this exercise is:

\begin{enumerate}
    \item Use Python to load/read data

    \item Use Numpy to manipulate data

    \item Use matplotlib to illustrate data

    \item Use string manipulation for tables

    % TODO \item Save Numpy data

\end{enumerate}


Note to remember.
If you sit with some data and you do not know how to plot the specific plot, then take a look at
\href{http://matplotlib.org/1.3.1/gallery.html}{matplotlib.org/1.3.1/gallery.html} for inspiration.



\subsection{Changing the look of matplotlib}






\newpage
\section{Exercises}

Todays exercises will each be based on different sets of data which requires different representation. 

<<<<<<< HEAD

\subsection{Dissociation Energy of Water Dimer}
=======
\subsection{Dissociation energy of water dimer}
>>>>>>> 4a61bd765af31248fdd71bcf44930015acb8b394

distance of the hydrogen bond is defined as the distance between the oxygen and the hydrogen

\begin{enumerate}

    \item Convert the distance from A.U. to \AA ngstroem.

    \item Convert the energy to kJoule/mol. Plot the result

    \item Convert the energy to kcal/mol. Plot the result.

\end{enumerate}



\newpage
\subsection{Binding free energies}

\begin{enumerate}
    \item Scatterplot

    \item Correlation factor

    \item RMSD

    \item Print table

\end{enumerate}


\subsection{Random precision errors in assigned chemical shifts}

When assigning measured chemical shifts of a protein to their respective amino-acids you will have to match the chemical shift measured by one experiment with another.
Due to experimental error these values are not exactly the same, even though they should be in theory.
Because of this it can be difficult to be sure that the two matched chemical shifts actually origin from the same amino-acid.
To avoid making assignment errors it is thus informative to know how well the measured chemical shifts 'should' match.

The file \code{chemical\_shift\_errors.txt} obtained from the course website contains all differences (or errors) in assigned chemical shifts of all amide protons for a single protein in a single column format.


\begin{enumerate}[resume]

    \item Load the file containing the data and store it a variable.

    \item Plot a histogram of the data using Matplotlib.

\end{enumerate}

When plotting a histogram you can select the number of bins to present the data with by giving the argument \code{bins=10}. (10 is the default value in Matplotlib).
If you try changing the number of bins you can severely affect how the data 'looks', especially if your number of datapoints are relatively low.
% put this in since all this shouldn't take long for the students to do.
The Freedman-Diaconis Rule can be used to select the number of bins automatically.
The following code takes as argument the data and returns the optimal number of bins according to the Freedman-Diaconis Rule.

\begin{lstlisting}
def bins(data):
    data.sort()
    n=len(data)
    width = 2*(data[3*n/4]-data[n/4])*n**(-1./3)
    return int((data[-1]-data[0])/width)
\end{lstlisting}

\begin{enumerate}[resume]

    \item Plot the histogram again using the Freedman-Diaconis Rule to select number of bins.

\end{enumerate}

If these errors are completely random, they should approximately follow a normal distribution (also known as a Gaussian distribution).
We can use the module \code{scipy.stats} to fit a distribution to a dataset. Import this in your program as follows:

\begin{lstlisting}
import scipy.stats as ss
\end{lstlisting}

We will begin by fitting a normal distribution to our data.
The command \code{ss.norm.fit(data)} returns the mean and standard deviation that best describes the data.
To draw this curve we will need a set of $x$ and $y$-values that cover our data range.

\begin{enumerate}[resume]

    \item Use \code{np.arange()} together with the \code{max()} and \code{min()} functions to generate $x$-values that range from the lowest data point to the highest in steps of \code{1e\-3}. Store these in a variable called \code{x}.

\end{enumerate}

The function \code{ss.norm.pdf(x, param\_1, param\_2)} returns the probability densities for a normal distribution for all values of \code{x}.
\code{param\_1} and \code{param\_2} are the mean and standard deviation you obtained from the fit.

Fitting more complicated distributions will return more than two parameters.
To avoid having to adjust the number of arguments for each distribution you look at, the following works for every distribution in the \code{scipy.stats} package:

\begin{lstlisting}
parameters = norm.fit(data)
y = norm.pdf(x, *parameters)
\end{lstlisting}


\begin{enumerate}[resume]
>>>>>>> 4a61bd765af31248fdd71bcf44930015acb8b394

    \item Fit a normal distribution to your data.

    \item Try to plot the fitted distribution together with the histogram. \emph{Hint!} Use \code{normed=1} as argument to your histogram.

    \item Try fitting other popular distributions such as the Cauchy/Lorenz distribution \code{ss.cauchy} or the Student's t-distribution \code{ss.t}.

\end{enumerate}

\begin{lstlisting}
\table{}

\end{lstlisting}



\newpage
\subsection{Proton transfer / reaction path}


\newpage
\subsection{Lars Fitting}


\newpage
\subsection{some kind of 3d plot}


\subsection{Protein structure determination}

Not all protein structures can easily be determined experimentally.
These kinds of proteins will often have their structure determined by simulation where a force-field is used to describe how the atoms interact with each other.
If a good force-field is used, the correct (also called native) structure should correspond to the lowest energy of the force-field.
When developing new force-fields you want to see how well the energy correlate with the deviation from the native structure.

The file \code{rmsd\_energy\_unfolded.txt} obtained from the course website, contains the energies, in units of $kcal/(mol\cdot RT)$ at $300K$, of several proposed structures as well as the atomic root mean square deviation (rmsd) in Å from the native structure.
You will never get an rmsd of zero for structures proposed by simulation, since the force-field will never fully describe all interactions correctly, however rmsds under 3-5 Å is usually considered accurate.

\emph{Note:} The first column contain the rmsds and the second one the energies.


\begin{enumerate}[start=1]

    \item Load the datafile in Python and put the rmsds in one list, and the energies in another.

    \item Plot the rmsds vs. energies using small black dots. Does the force-field used seem to be good in this case?

\end{enumerate}

Since the local energi minimum is not exactly at 0 Å it can be hard to know if you have achieved the correct structure and the non-zero rmsd stems from the force-field, or if an incorrect structure is found.
To test this a second simulation is usually run starting from the native structure to what the structure relaxes to with the given force-field.


\begin{enumerate}[resume]

    \item Download and load the file \code{rmsd\_energy\_native.txt}.

    \item Plot this together with the data from before, using a red colour. \emph{Note!} Change the border colour of the dots to red as well.

    \item Based on this second simulation, would you say that the force-field performs well?

\end{enumerate}

% ***************************************************
% END DOCUMENT
% ***************************************************

\end{document}
