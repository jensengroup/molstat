\documentclass{article}

\usepackage[utf8]{inputenc}

% Packages
\usepackage{amsmath,amssymb}
\usepackage{bm}% boldmath
\usepackage{listings} % Code block (source code) \begin{lstlisting} 
\usepackage{natbib}
\usepackage{graphicx}
\usepackage{lmodern}
\usepackage[usenames,dvipsnames,svgnames,table]{xcolor}
\usepackage[textwidth=16cm,textheight=23cm]{geometry}

%\usepackage{inconsolata} % New monospace font

% URL
\usepackage{url}
\usepackage[colorlinks=true, a4paper=true, pdfstartview=FitV, linkcolor=blue, citecolor=blue, urlcolor=blue]{hyperref}

% Figures
\usepackage[font=small, labelfont=bf]{caption}
\usepackage{subfig} % Subfigures. Uses \subfloat[captions text]{figure}

% Tables
\usepackage{booktabs}   % Allows the use of \toprule, \midrule and \bottomrule in tables for horizontal lines
\newcommand{\ra}[1]{\renewcommand{\arraystretch}{#1}} % spaces in tables

% Itemize
\usepackage{enumitem}

% Commands
%\newcommand{\code}[1]{\texttt{#1}} % \code{inline code}
\newcommand{\code}[1]{{\small\ttfamily #1}} % \code{inline code}
\newcommand{\expval}[1]{\langle #1 \rangle} %
\renewcommand{\theequation}{\arabic{section}.\arabic{equation}} % Book format equation
\renewcommand{\thefigure}{\arabic{section}.\arabic{figure}} % Book format figure
\renewcommand{\vec}[1]{{\bf #1}} % Lars likes this better than arrow

% Set page attribution
\setlength{\parindent}{0pt}


% PSTRICKS
\usepackage{pstricks,pst-node,pst-tree} % includes graph additions
\usepackage{pst-pdf} % Compiles the pictures
\usepackage{pst-node}
\usepackage{pst-plot}
\usepackage{pst-3dplot}
%\usepackage{pstricks-add,babel}




\lstset{
language=Python,                        % Code langugage
commentstyle=\color{gray},              % Comments font
basicstyle=\small\ttfamily,             % Code font, Examples: \footnotesize, \ttfamily
keywordstyle=\bfseries\color{blue},
stringstyle=\color{orange},
numbers=left,                           % Line nums position
numberstyle=\tiny,                      % Line-numbers fonts
stepnumber=1,                           % Step between two line-numbers
numbersep=5pt,                          % How far are line-numbers from code
frame=none,                             % A frame around the code
tabsize=4,                              % Default tab size
captionpos=b,                           % Caption-position = bottom
breaklines=true,                        % Automatic line breaking?
breakatwhitespace=false,                % Automatic breaks only at whitespace?
showspaces=false,                       % Dont make spaces visible
showstringspaces=false,                 % Dont make spaces visible in strings
showtabs=false,                         % Dont make tabls visible
belowskip=8pt,
morekeywords={range, xrange, as},
% backgroundcolor=\color{gray}
% emph={[2]root,base}
% morekeywords={one,two,three,four,five,six,seven,eight,
}


%commentstyle=\color{gray},              % Comments font
%basicstyle=\small,                      % Code font, Examples: \footnotesize, \ttfamily



%basicstyle=\footnotesize\ttfamily,
%keywordstyle=\bfseries\color{green!40!black},
%commentstyle=\itshape\color{purple!40!black},
%identifierstyle=\color{blue},
%stringstyle=\color{orange},







% ***************************************************
% HEADER INFORMATION

\title{Genetic Algorithm}
\author{Molecular Statistic}
\date{2013}

% ***************************************************

\begin{document}


% ***************************************************
% BEGIN DOCUMENT
% ***************************************************


\maketitle

\section{Introduction}

This is the exam project in the course molecular statistics which involves the genetic algorithm
for energy minimization.
The purpose of this exam
project is to do a energy
minimization of long alkane molecules using an algorithm called the genetic algorithm which
is a flavor of the general Monte-Carlo approach with a force field package called MMTK.

\subsection{Force Field and Dihedral angle}

For this project we will be using force field theory to do
the energy minimization, which means the energy of a curtain
molecule configuration is calculated by eq. 2.5 from Jensen;

\begin{align}
    E_{molecule}
    &= \sum^{bonds}_i k_i (r_i - r_{i,e})^2
    + \sum^{angles}_i k_i ( \sigma_i - \sigma_{i,e})^2
    + \sum^{dihedrals}_i V_i [1 \pm \cos (n_i \omega_i) ] \nonumber \\
    &+ \sum^{atom pairs}_{i>j} \left ( -\frac{A_iA_j}{r^6_{ij}} + \frac{B_iB_j}{r^12_{ij}} + \frac{q_iq_j}{r_{ij}} \right )
    \label{eq:amber}
\end{align}


The force field calculation will be done by the Amber 94 forcefield
using a python package called MMTK.
%ASC - Delete sentence: From eq. \ref{eq:amber} we will be focusing on the dihedral part of the energy.

\begin{figure}[htb!]
  % from
  % http://upload.wikimedia.org/wikipedia/commons/4/42/Bond_dihedral_angle.png

	\centering
	\psset{xunit=1cm,yunit=1cm}
	\begin{pspicture}(-2,-3)(4,3)
		\psframe(-2,-3)(4,3)
		%\rput(0,0){O}
    \psline{->}(0.0,0.0)(2.0,0.0)
    \rput{60.0}(0.0,0.0){
      \psline{->}(-2.0,0.0)(0.0,0.0)
      \pscircle(-2.0,0.0){0.3}
    }
    \rput{60.0}(2.0,0.0){
      \psline{->}(0.0,0.0)(2.0,0.0)
      \pscircle(2.0,0.0){0.3}
    }
    % Circles
    \pscircle(0.0,0.0){0.3}
    \pscircle(2.0,0.0){0.3}
    \rput(3.0,0.0){.}
    \psellipse[linestyle=dashed,dash=2pt](3.0,0.0)(0.2,1.7)
    \psellipse[linestyle=dashed,dash=1pt](1.0,0.0)(0.15,0.6)
    \psframe*[linecolor=white](0.84,0.1)(1.16,0.7)
    \parabola{->}(0.85,0.0)(1.0,0.6)
    \rput(1.15,0.8){$\omega_i$}
    \rput(-1.5, -1.4){A}
    \rput(-0.4, 0.4){B}
    \rput(2.0, -0.6){C}
    \rput(2.6, 2.1){D}
	\end{pspicture}
  \label{fig:dihedral}
  \caption{
    Illustrating the dihedral angle $\omega_i$ between 3 bonds and 4 carbon centers, A, B, C and D.
  }

\end{figure}

In geometry, a dihedral or torsion angle is the angle between two planes, and
in our alkane case it is the angle between two CH$_2$ groups, as seen in figure \ref{fig:dihedral}.
We want to change the angles in such a way that it finds the minimal energy
from a dihedral configuration $\Omega$.
The configuration $\Omega$ is then defined as a vector of
the systems dihedral angles $\omega_i$ beween each of the successive CH$_2$ groups,
$\Omega_\beta = \{ \omega_1, \omega_2, ... , \omega_N \}$.
Each $\omega_i$ is assigned an initial random value taken
from a uniform distribution
from 0 to 360 degrees.
These dihedral angles can be seen as the genetic code for the alkanes and by modifying
these angles we can minimize the molecules geometry with respect to the total molecule energy..


\subsection{Genetic Algorithm}

The genetic algorithm is an optimization algorithm that is based on genetic inheritance, 
hence the name. 
The genetic algorithm is an efficient algorithm to search through conformational space and locate minima. 
Here we employ the genetic algorithm to try and locate the lowest energy conformation for an alkane.

As described before we define the geometry of an alkane chain through its
dihedral angles $\omega_i$ between each of the successive CH$_2$ groups,
and so by modifying these angles we change the structure and thereby the energy.\\


We start out by defining a state vector $\Omega_\beta$ as
$\Omega_\beta = \{ \omega_1, \omega_2, ... , \omega_N \}$
where we have $N$ dihedral genetic angels.
Each $\omega_i$ is assigned an initial random value taken
from a uniform distribution
from 0 to 360 degrees. This can easily be done using
numpy, for $N$ dihedral angles;

\begin{lstlisting}
state_vector = np.random.uniform(0.0, 360.0, N)
\end{lstlisting}

We then create a set of $K$ state vectors.
For each of these state vectors,
we define a corresponding energy of the molecule $E_\beta(\Omega_\beta)$.
The energy is obtained after a short
{\em Conjugate Gradient minimization}\footnote{This will be explained in the setup section}.
From now on, these $K$ state vectors are referred to as \emph{parents}.\\

The parents are then changed as a result from the following algorithm steps;

\begin{enumerate}
  \item {\bf Mating.}\newline
    Two parents $\Omega_\beta$ and $\Omega_\alpha$ of length $N$
    can combined to give birth to two children with the "genome" 
    of the parents represented by the its state vector.
    The state vector $\Omega_\beta$ is then being split from the first
    $M$ elements into the first child, and ($M-N$) into the next child.
    Same procedure for parent $\Omega_\alpha$.
    The cut index $M$ is to be determined randomly.

  \item {\bf Mutation.}\newline
    After the children has been born, sometimes, a random mutation is
    inserted in their 'genome'.
    This means randomly select one of the $N$ angles $\omega_i$
    and assign it a new random value, to a probablity
    of a mutation rate, \code{mutatation\_rate}.

  \item {\bf Evolution.} \newline
    Survival of the fittest.
    The energy of the formed child is evaluated like the parents, minimize
    the geometry of its state vector.
    If the energy if the child is lower than one of the parents, then the child
    is kept and the parent with the larger energy is removed,
    where the child takes its parents place.
    If the energy of the child is larger than a parent, there is still a probability $P$
    that it survives the evolution step.      
    This is where the Monte Carlo enters the algorithm.
    $P$ is proportional to the Boltzmann factor, i.e.
    \begin{align}
      P = e^{-\Delta E /(k_\mathrm{B}T)}
    \end{align}

    where $\Delta E$ is the energy difference between the child and the parent,
    and $T$ is the temperature of the system. In the following we set $k_\mathrm{B} = 1$, 
    so all temperatures are in units of  $k_\mathrm{B}$.\\

    When {\em one} child has replaced a parent,
    we consider this to be the end of the generation.\footnote{This is where the \code{break} command comes in handy in a for-loop}
    This means that no more mating of parents 
    should be carried out and we start over. 

\end{enumerate}


\newpage
\section{Assignment}

The main task of this assignment is to implement
the genetic algorithm, and minimize the structure of 
three alcane chains,
C$_{10}$H$_{22}$,
C$_{20}$H$_{42}$ and
C$_{40}$H$_{82}$,

\subsection{Setup}

To successfully do this assignment you need to install
some software dependencies on your computer.
This means downloading \code{install\_mmtk.sh} from
the project folder (along with rest of the files).
Use \code{cd} to find the file and run the installation script like this;


\begin{lstlisting}
sudo sh install_mmtk.sh
\end{lstlisting}

in the terminal.\\

Remember to download the rest of the files from the
absalon page.


\subsection{Examples}

Make sure you have
\code{minimizers.py},
\code{molecule.py},
\code{dbutane.py},
\code{ddecane.py},
\code{sdicosane.py} and
\code{dtetracontane.py}
in the same folder as the python
file you working in. To use the functionality 
provided by these packages you import them
in the beginning of your python file:

\begin{lstlisting}
from molecule import Molecule
from minimizers import conjugateGradient
from sdbutane import setDihedral
\end{lstlisting}

Where \code{sdbutane} is imported when working on
the butane alkane chain.\\

To do a energy calculation of a random dihedral state of Butane,
you use the modules in the following way.

\begin{lstlisting}
m = Molecule('butane')
no_dihedral = 1
dihedral_list = np.random.uniform(0.0, 360.0, no_dihedral)

setDihedral(m, dihedral_list)
conjugateGradient(m)
energy =  m.getEnergy()
\end{lstlisting}

The above code might seem strange as it is using 'object orientated'
coding which we did not cover in this course.
On line 1 the molecule \code{m} is defined,
and on line 5 the dihedral angles of the molecule is set
which energy is calculated on line 6 and 7.\\

The function \code{conjugateGradient} is where the
forcefield minimization happens. This function finds
the nearest local minimal a stops, where after
it is possible to get the energy from the molecule.
You need to run the functions in that order for it to work.\\

The above code is the only code you need to solve this
assignment.
The rest can be implemented by you.\\

If you want to see how a molecule looks, use code:
\begin{lstlisting}
m.saveXYZ("my_molecule.xyz")
\end{lstlisting}
You can open the my\_molecule.xyz file in a program called Avogadro by
typing "avogadro my\_molecule.xyz" in the command line. See the example
in file example\_savebutane.py.

\newpage

\subsection{Simulation Setup}

Before you finish the actual simulation you need to create
the functions that simulates the system.
To make it easy to start we have split the setup into
small manageable steps.

\begin{enumerate}
  \item Use the code from \code{example\_butane.py} and familiarise yourself with it.

  \item Create a list of angles from 0.0 to 180.0 degrees and calculate the energy
  of Butane for each angle. Plot the result.

  \item Implement the following optimization algorithm, called \textit{Greedy optimization}:
    \begin{enumerate}
      \item Create a random list of dihedral angles
      \item Create a integer \code{no\_generations} which represents
        how long the algorithm will go on.
      \item Create a for-loop and loop over the no of generations.
        For each loop generate a random dihedral state and calculate the energy.
      \item If the energy is lower than the previous, save the new energy and the new state,
        otherwise discard new state the away and continue the search.
    \end{enumerate}

  \item Create a function that takes \code{m} and \code{dihedral\_list} as
    parameters and returns the energy of the configuration.

  \item Create a function that takes the parameters
    \code{parent\_alpha},
    \code{parent\_beta} and
    \code{mutation\_rate}.
    Have this function mate the two parents and
    create two children based on the genetic algorithm
    step 1 and 2. Use numpy's \code{ranint} generate a random cut index $M$

\begin{lstlisting}
m = np.random.ranint(0, N)
\end{lstlisting}

  \item Create two lists, one for the containing the state vectors
    and one containing the energy related to the state vector.
    Fill them up with \code{no\_parents} of random dihedral states,
    and calculate the energy for each.

  \item Finish the algorithm by creating a for-loop and loop over
    number of generations defined,
    mating each parent pair, selecting what parent and what child
    survives based on the Genetic algorithm.

\end{enumerate}


\subsection{Simulations}

Now that you have a working simulation, it is time to
do the actual simulations:

\begin{itemize}
    \item {\bf Simulation 1}\newline
      Minimize
      C$_{10}$H$_{22}$,
      C$_{20}$H$_{42}$ and
      C$_{40}$H$_{82}$
      and plot the mean energy for
      each generation.

    \item {\bf Simulation 2}\newline
      Check for correlation between temperature $T$ and
      the energy found after $G$ generation for each molecule.
      Check for temperatures between 0.5 and 10.0.

    \item {\bf Simulation 3}\newline
      Check for correlation between mutation rate in the interval [0:0.5] and
      the energy found after $G$ generation for each molecule.

    \item {\bf Simulation 4}\newline
      Compare the Greedy and Genetic algorithm by
      plotting the energy vs generation.

    \item {\bf Simulation 5}\newline
      Check the population size $K$ of parents.
      Plot the number of generations it took to converge
      as a function of the number of $K$ parents.

    \item {\bf Simulation 6}\newline
      Make the Genetic Algorithm better.
      You have to come up with one changes to the mating
      and mutation routine.
      Change the algorithm to make your own personal
      minimization algorithm. Document the results.


\end{itemize}





% ***************************************************
% END DOCUMENT
% ***************************************************

\end{document}

