\documentclass{article}

\usepackage[utf8]{inputenc}

% Packages
\usepackage{amsmath,amssymb}
\usepackage{bm}% boldmath
\usepackage{listings} % Code block (source code) \begin{lstlisting} 
\usepackage{natbib}
\usepackage{graphicx}
\usepackage{lmodern}
\usepackage[usenames,dvipsnames,svgnames,table]{xcolor}
\usepackage[textwidth=16cm,textheight=23cm]{geometry}

%\usepackage{inconsolata} % New monospace font

% URL
\usepackage{url}
\usepackage[colorlinks=true, a4paper=true, pdfstartview=FitV, linkcolor=blue, citecolor=blue, urlcolor=blue]{hyperref}

% Figures
\usepackage[font=small, labelfont=bf]{caption}
\usepackage{subfig} % Subfigures. Uses \subfloat[captions text]{figure}

% Tables
\usepackage{booktabs}   % Allows the use of \toprule, \midrule and \bottomrule in tables for horizontal lines
\newcommand{\ra}[1]{\renewcommand{\arraystretch}{#1}} % spaces in tables

% Itemize
\usepackage{enumitem}

% Commands
%\newcommand{\code}[1]{\texttt{#1}} % \code{inline code}
\newcommand{\code}[1]{{\small\ttfamily #1}} % \code{inline code}
\newcommand{\expval}[1]{\langle #1 \rangle} %
\renewcommand{\theequation}{\arabic{section}.\arabic{equation}} % Book format equation
\renewcommand{\thefigure}{\arabic{section}.\arabic{figure}} % Book format figure
\renewcommand{\vec}[1]{{\bf #1}} % Lars likes this better than arrow

% Set page attribution
\setlength{\parindent}{0pt}


% PSTRICKS
\usepackage{pstricks,pst-node,pst-tree} % includes graph additions
\usepackage{pst-pdf} % Compiles the pictures
\usepackage{pst-node}
\usepackage{pst-plot}
\usepackage{pst-3dplot}
%\usepackage{pstricks-add,babel}




\lstset{
language=Python,                        % Code langugage
commentstyle=\color{gray},              % Comments font
basicstyle=\small\ttfamily,             % Code font, Examples: \footnotesize, \ttfamily
keywordstyle=\bfseries\color{blue},
stringstyle=\color{orange},
numbers=left,                           % Line nums position
numberstyle=\tiny,                      % Line-numbers fonts
stepnumber=1,                           % Step between two line-numbers
numbersep=5pt,                          % How far are line-numbers from code
frame=none,                             % A frame around the code
tabsize=4,                              % Default tab size
captionpos=b,                           % Caption-position = bottom
breaklines=true,                        % Automatic line breaking?
breakatwhitespace=false,                % Automatic breaks only at whitespace?
showspaces=false,                       % Dont make spaces visible
showstringspaces=false,                 % Dont make spaces visible in strings
showtabs=false,                         % Dont make tabls visible
belowskip=8pt,
morekeywords={range, xrange, as},
% backgroundcolor=\color{gray}
% emph={[2]root,base}
% morekeywords={one,two,three,four,five,six,seven,eight,
}


%commentstyle=\color{gray},              % Comments font
%basicstyle=\small,                      % Code font, Examples: \footnotesize, \ttfamily



%basicstyle=\footnotesize\ttfamily,
%keywordstyle=\bfseries\color{green!40!black},
%commentstyle=\itshape\color{purple!40!black},
%identifierstyle=\color{blue},
%stringstyle=\color{orange},







% ***************************************************
% HEADER INFORMATION

\title{Genetic Algorithm}
\author{Molecular Statistics}
\date{2014}

% ***************************************************

\begin{document}


% ***************************************************
% BEGIN DOCUMENT
% ***************************************************


\maketitle

\section{Introduction}

The genetic algorithm is a general minimization algorithm which in this project is used to minimize the energy of alkane molecules, in the flavor of the general Monte-Carlo approach.
The energy is calculated using the force field MMFF94 as implemented in the software package OpenBabel.

\subsection{Force field and dihedral angle}

We will be using a classic force field, and not quantum mechanics to simulate the molecule.
The energy for the molecule in a certain configuration is calculated as

\begin{align}
    E_\mathrm{configuration}
    &= \sum^\mathrm{bonds}_i k_i (r_i - r_{i,e})^2
    + \sum^\mathrm{angles}_i k_i ( \sigma_i - \sigma_{i,e})^2
    + \sum^\mathrm{dihedrals}_i V_i [1 \pm \cos (n_i \omega_i) ] \nonumber \\
    &+ \sum^\mathrm{atom pairs}_{i>j} \left ( -\frac{A_iA_j}{r^6_{ij}} + \frac{B_iB_j}{r^12_{ij}} + \frac{q_iq_j}{r_{ij}} \right )
    \label{eq:amber}
\end{align}




\begin{figure}[htb!]

	\centering
	\psset{xunit=1cm,yunit=1cm}
	\begin{pspicture}(-2,-3)(4,3)
		\psframe(-2,-3)(4,3)
		%\rput(0,0){O}
    \psline{->}(0.0,0.0)(2.0,0.0)
    \rput{60.0}(0.0,0.0){
      \psline{->}(-2.0,0.0)(0.0,0.0)
      \pscircle(-2.0,0.0){0.3}
    }
    \rput{60.0}(2.0,0.0){
      \psline{->}(0.0,0.0)(2.0,0.0)
      \pscircle(2.0,0.0){0.3}
    }
    % Circles
    \pscircle(0.0,0.0){0.3}
    \pscircle(2.0,0.0){0.3}
    \rput(3.0,0.0){.}
    \psellipse[linestyle=dashed,dash=2pt](3.0,0.0)(0.2,1.7)
    \psellipse[linestyle=dashed,dash=1pt](1.0,0.0)(0.15,0.6)
    \psframe*[linecolor=white](0.84,0.1)(1.16,0.7)
    \parabola{->}(0.85,0.0)(1.0,0.6)
    \rput(1.15,0.8){$\omega$}
    \rput(-1.5, -1.4){A}
    \rput(-0.4, 0.4){B}
    \rput(2.0, -0.6){C}
    \rput(2.6, 2.1){D}
	\end{pspicture}

    \caption{
        Illustrating the dihedral angle $\omega$ between 3 bonds and 4 carbon centers, A, B, C and D.
    }
    \label{fig:dihedral}

\end{figure}

To simplify things, we will only be focusing on the dihedral part of the energy expression.
Geometrically, a dihedral (or torsion) angle is the angle between two planes. In our alkane case it is the angle between two CH$_2$ groups (or rather the plane created by carbon atoms ABC and the plane created by atoms BCD), as illustrated in Figure \ref{fig:dihedral}.

We want to optimize the dihedral angles in such a way that the lowest configuration energy is found based on the dihedral configuration $\Omega$.
The configuration $\Omega$ is defined as a set of the systems dihedral angles $\omega_i$ between each of the successive CH$_2$ groups, $\Omega = \{ \omega_1, \omega_2, ... , \omega_N \}$.\\

In the beginning each $\omega_i$ is assigned an initial random value, taken from an uniform distribution, between 0 and 360 degrees.
The task in hand is then to use the genetic algorithm and modify these $\omega_i$ angles such that the total molecular energy will be minimized in respect to the geometrical configuration.

\subsection{Genetic Algorithm}

The genetic algorithm is an optimization algorithm that is based on genetic inheritance.
The algorithm is efficient in searching through conformational space and locate minima.
Here we employ the genetic algorithm to try and locate the lowest energy conformation for an alkane.\\

As described before we define the geometry of an alkane chain through its dihedral angles $\omega_i$ between each of the successive CH$_2$ groups, and so by modifying these angles we change the structure and thereby the energy.\\

We start out by defining a state vector $\Omega$ as $\Omega = \{ \omega_1, \omega_2, ... , \omega_N \}$ where we have $N$ dihedral angles.
Each $\omega_i$ is assigned an initial random value from an uniform distribution from 0 to 360 degrees.
%
% This can easily be done using numpy, for $N$ dihedral angles
% 
% \begin{lstlisting}
% Omega_beta = np.random.uniform(0.0, 360.0, N)
% \end{lstlisting}
%
We then create a set of $K$ state vectors.
For each of these state vectors, we define a corresponding energy of the molecule $E(\Omega_K)$.
The energy is obtained via the force field.
From now on, these $K$ state vectors are referred to as \emph{parents}.\\

The parents are then modified as a result from the following algorithm steps:

\begin{enumerate}

    \item {\bf Mating.}\newline
        Two parents $\Omega_\alpha$ and $\Omega_\beta$ of length $N$ can combine to give birth to two children, each receiving a part of the 'genome' of the parent.
        The first child will inherit the first $M$ elements from parent $\Omega_\alpha$ and the remaining $M-N$ elements from parent $\Omega_\beta$, and vice versa for the second child.
        The cut index $M$ is to be determined randomly.

    \item {\bf Mutation.}\newline
        After the children has been born, sometimes a random mutation is inserted in their genome.
        This means randomly select one of the $N$ angles $\omega_i$ and assigning it a new random value, with a probability \code{mutation\_rate}.

    \item {\bf Evolution.} \newline
        Survival of the fittest.
        If the energy of a child is lower than one of the parents, then the child is kept and the parent with the larger energy is removed (the child takes its parents place).
        If the energy of the child is larger than a parent, there is still a probability $P$ that it survives the evolution step.           The probability $P$ is given by the Monte Carlo Metropolis Hastings acceptance criterion, i.e.
        \begin{align}
        P = e^{-\Delta E /(k_\mathrm{B}T)}
        \end{align}

        where $\Delta E$ is the energy difference between the child and the parent, and $T$ is the temperature of the system.
        In the following we set $k_\mathrm{B} = 1$ so all temperatures are in units of  $k_\mathrm{B}$.\\
        When {\em one} child has replaced a parent, we consider this to be the end of the generation.\footnote{A big hint here would to look up the \code{break} command for loops}

\end{enumerate}


\newpage
\section{Assignment}

The main task of this assignment is to implement the genetic algorithm and minimize the energy of three alkane chains,
C$_{10}$H$_{22}$,
C$_{20}$H$_{42}$ and
C$_{40}$H$_{82}$.

\subsection{Setup}

To successfully do this assignment you need to install some software dependencies on your computer.
This is done by writing the follow command in the terminal (remember that the molbox password is \emph{science}).

\begin{lstlisting}
sudo apt-get install openbabel python-openbabel
\end{lstlisting}

\subsection{Examples}

You will be using the module \code{molecule.py} which in turn uses the openbabel package.
The module is imported the usual way and can be found on the course website.

\begin{lstlisting}
import molecule as mol
\end{lstlisting}

Please also download \code{butane\_example.py} from the course website and inspect the code.\\

There are three functions you will be using from the module.

\begin{lstlisting}
mol.generate_alkane(N)   # Generates an alkane chain of N length.
mol.set_dihedral(angles) # Sets the dihedral angles of the molecule. Requires angles as a list.
mol.get_energy()         # Calculates the current energy of the molecule. Returns energy.
\end{lstlisting}

The molecule and the coordinates itself is defined and stored in the \code{molecule} module.
To save the structure you can use the following function

\begin{lstlisting}
mol.save_molecule('molecule_name.xyz')
\end{lstlisting}

which will save the structure in XYZ format.
You can use the program Avogadro or JMol to open and see the structure.

\newpage

\subsection{Simulation Setup}

To make it easy to get started we have split the setup into small manageable steps.

\begin{enumerate}
    \item Use the code from \code{butane\_example.py} and familiarize yourself with it.

    \item Create a list of angles from 0.0 to 180.0 degrees and calculate the energy of butane for each angle (you only have to generate the alkane once)
    Plot the result.

    \item Implement the following optimization algorithm, called \textbf{Greedy optimization}:
        \begin{enumerate}
            \item Create a random dihedral state.
            \item Create an integer \code{no\_generations} which represents how long the algorithm will go on.
            \item Create a for-loop and loop over the no. of generations. 
            For each loop generate a random dihedral state and calculate the energy.
            \item If the energy is lower than the previous generation, save the new energy and the new state, otherwise discard the new state and continue the search.
        \end{enumerate}

    \item Use the Greedy algorithm to optimize alkane structures.

    \item Create a function that takes \code{angles} as a parameter and returns the energy.

    \item Create a function that takes the parameters
    \code{parent\_alpha},
    \code{parent\_beta} and
    \code{mutation\_rate}.
    Have this function mate the two parents and create two children based on the genetic algorithm step 1 and 2.
    Use numpy's \code{randint} generate a random cut index $M$.

\begin{lstlisting}
M = np.random.randint(0, N)
\end{lstlisting}

    \item Create two lists, one containing the state vectors
    and one containing the energy of the corresponding state vector.
    Fill them up with \code{no\_parents} of random dihedral states,
    and calculate the energy for each.

    \item Finish the algorithm by creating a for-loop and loop over the number of generations defined. For each parent pair select which parents and children survives based of the genetic algorithm.

\end{enumerate}


\subsection{Simulations}

\begin{itemize}
    \item {\bf Simulation 1}\newline
      Minimize
      C$_{10}$H$_{22}$,
      C$_{20}$H$_{42}$ and
      C$_{40}$H$_{82}$
      and plot the mean and minimum energy for
      each generation.

    \item {\bf Simulation 2}\newline
      Check for correlation between temperature $T$ and the energy found after $G$ generations for each molecule.
      Check for temperatures between 0.5 and 10.0.

    \item {\bf Simulation 3}\newline
      Check for correlation between mutation rate in the interval [0:0.5] and the energy found after $G$ generation for each molecule.

    \item {\bf Simulation 4}\newline
      Compare the Greedy and Genetic algorithm by plotting the min energy vs generation.

    \item {\bf Simulation 5}\newline
      Check the population size $K$ of parents.
      Plot the number of generations it took to converge as a function of the number of $K$ parents.

    \item {\bf Simulation 6}\newline
      Make the Genetic Algorithm better.
      You have to come up with changes to the mating and mutation routine.
      Change the algorithm to make your own personal minimization algorithm.
      Document the results.


\end{itemize}



% ***************************************************
% END DOCUMENT
% ***************************************************

\end{document}

