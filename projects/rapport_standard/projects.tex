\documentclass{article}
\usepackage[utf8]{inputenc}

% Packages
\usepackage{amsmath,amssymb}
\usepackage{bm}% boldmath
\usepackage{listings} % Code block (source code) \begin{lstlisting} 
\usepackage{natbib}
\usepackage{graphicx}
\usepackage{lmodern}
\usepackage[usenames,dvipsnames,svgnames,table]{xcolor}
\usepackage[textwidth=16cm,textheight=23cm]{geometry}

% Commands
\newcommand{\code}[1]{\texttt{#1}} % \code{inline code}
\newcommand{\expval}[1]{\langle #1 \rangle} %
\renewcommand{\theequation}{\arabic{section}.\arabic{equation}} % Book format equation
\renewcommand{\thefigure}{\arabic{section}.\arabic{figure}} % Book format figure

% Set page attribution
\setlength{\parindent}{0pt}


% PSTRICKS
\usepackage{pstricks,pst-node,pst-tree} % includes graph additions
\usepackage{pst-pdf}	% Compiles the pictures
\usepackage{pst-node}
\usepackage{pst-plot}
\usepackage{pst-3dplot}
%\usepackage{pstricks-add,babel}


% ***************************************************
% HEADER INFORMATION

\title{Report Standard}
\author{Molecular Statistics}
\date{}

% ***************************************************

\begin{document}


% ***************************************************
% BEGIN DOCUMENT
% ***************************************************


\maketitle

\section*{Project Report}

The report is done individually, should be written in either Danish or English and be between ca. 10 and no more than 20 pages long.
The code associated with the study should be uploaded on the course website or emailed to the instructor.\\

The report should be written with enough theoretical and technical background so any of your colleagues can understand what you have done, and be able to reproduce your results.\\

Just like reports you have written before, the project report should include the following sections

\begin{itemize}

    \item {\bf Title}

    \item {\bf Author Name}

    % Does this matter?
    % \item {\bf Keywords (4-5 words)}

    \item {\bf Abstract (Max. 200 characters)}
        An abstract is a brief {\bf summary } of a research article, thesis, review, conference proceeding or any in-depth analysis of a particular subject or discipline, and is often used to help the reader quickly ascertain the paper's purpose.


    \item {\bf Introduction and theory.}
    A theoretical introduction to the subject along with motivation for the work.
    This includes a description of the theory and algorithms, as well as figures that could help with the understanding of the subject.

    \item {\bf Implementation.}
    Comment on what the code does and how you implemented the code.
    Youy are welcome to use code examples.

    \item {\bf Results and discussion.}
    Presentation of your results (use figures, pictures and tables).
    Discussion of the results.
    Do the results behave as expect, are there outliers and 
    do you trust your results?

    \item {\bf Conclusion.}
    Summary of the most important findings.
    Perspective of your results.

    \begin{itemize}
        \item Did you achieve the goal described in the introduction?
        \item What would be the next step?
        \item Could one improve on your calculations?
    \end{itemize}

    \item {\bf References.}
    It is okay to refer to book chapters or websites.

\end{itemize}











% ***************************************************
% END DOCUMENT
% ***************************************************

\end{document}

