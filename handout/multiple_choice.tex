%\documentclass[answers]{exam} %shows the correct answer
\documentclass{exam}
\usepackage[utf8]{inputenc}

% Packages
\usepackage{amsmath,amssymb}
\usepackage{bm}% boldmath
\usepackage{listings} % Code block (source code) \begin{lstlisting} 
\usepackage{natbib}
\usepackage{graphicx}
\usepackage{lmodern}
\usepackage[usenames,dvipsnames,svgnames,table]{xcolor}
\usepackage[textwidth=16cm,textheight=23cm]{geometry}

%\usepackage{inconsolata} % New monospace font

% URL
\usepackage{url}
\usepackage[colorlinks=true, a4paper=true, pdfstartview=FitV, linkcolor=blue, citecolor=blue, urlcolor=blue]{hyperref}

% Figures
\usepackage[font=small, labelfont=bf]{caption}
\usepackage{subfig} % Subfigures. Uses \subfloat[captions text]{figure}

% Tables
\usepackage{booktabs}   % Allows the use of \toprule, \midrule and \bottomrule in tables for horizontal lines
\newcommand{\ra}[1]{\renewcommand{\arraystretch}{#1}} % spaces in tables

% Itemize
\usepackage{enumitem}

% Commands
%\newcommand{\code}[1]{\texttt{#1}} % \code{inline code}
\newcommand{\code}[1]{{\small\ttfamily #1}} % \code{inline code}
\newcommand{\expval}[1]{\langle #1 \rangle} %
\renewcommand{\theequation}{\arabic{section}.\arabic{equation}} % Book format equation
\renewcommand{\thefigure}{\arabic{section}.\arabic{figure}} % Book format figure
\renewcommand{\vec}[1]{{\bf #1}} % Lars likes this better than arrow

% Set page attribution
\setlength{\parindent}{0pt}


% PSTRICKS
\usepackage{pstricks,pst-node,pst-tree} % includes graph additions
\usepackage{pst-pdf} % Compiles the pictures
\usepackage{pst-node}
\usepackage{pst-plot}
\usepackage{pst-3dplot}
%\usepackage{pstricks-add,babel}




\lstset{
language=Python,                        % Code langugage
commentstyle=\color{gray},              % Comments font
basicstyle=\small\ttfamily,             % Code font, Examples: \footnotesize, \ttfamily
keywordstyle=\bfseries\color{blue},
stringstyle=\color{orange},
numbers=left,                           % Line nums position
numberstyle=\tiny,                      % Line-numbers fonts
stepnumber=1,                           % Step between two line-numbers
numbersep=5pt,                          % How far are line-numbers from code
frame=none,                             % A frame around the code
tabsize=4,                              % Default tab size
captionpos=b,                           % Caption-position = bottom
breaklines=true,                        % Automatic line breaking?
breakatwhitespace=false,                % Automatic breaks only at whitespace?
showspaces=false,                       % Dont make spaces visible
showstringspaces=false,                 % Dont make spaces visible in strings
showtabs=false,                         % Dont make tabls visible
belowskip=8pt,
morekeywords={range, xrange, as},
% backgroundcolor=\color{gray}
% emph={[2]root,base}
% morekeywords={one,two,three,four,five,six,seven,eight,
}


%commentstyle=\color{gray},              % Comments font
%basicstyle=\small,                      % Code font, Examples: \footnotesize, \ttfamily



%basicstyle=\footnotesize\ttfamily,
%keywordstyle=\bfseries\color{green!40!black},
%commentstyle=\itshape\color{purple!40!black},
%identifierstyle=\color{blue},
%stringstyle=\color{orange},







% ***************************************************
% HEADER INFORMATION

\title{Multiple choice}
\author{Molecular Statistics 2016}
\date{}

% ***************************************************

\begin{document}
\maketitle

\begin{center}
\fbox{\fbox{\parbox{5.5in}{\centering
Check off the circle to the left of the answer you think is correct.}}}
\end{center}
\vspace{0.1in}
\makebox[\textwidth]{Name:\enspace\hrulefill}
\vspace{0.2in}
\makebox[\textwidth]{KU-username:\enspace\hrulefill}


\begin{questions}
\question What is the output of \code{L[1]} if \code{L = [1,2,3]}?
\begin{checkboxes}
\choice 2,3
\CorrectChoice 2
\choice 3
\choice 1
\end{checkboxes}
\vspace{1cm}

\question If \code{list = [1,2,3]}, which of the following woud then gives \code{list = [1,2,3,4]}?
\begin{checkboxes}
\CorrectChoice list.append(4)
\choice list.sort()
\choice list + [4]
\choice list.remove(4)
\end{checkboxes}
\vspace{1cm}


\question Which of the following operator in python performs exponential (power) calculation on operands?
\begin{checkboxes}
\CorrectChoice **
\choice \%
\choice is
\choice +=
\end{checkboxes}
\vspace{1cm}


\question What is the type of \code{2.0}?
\begin{checkboxes}
\choice int
\CorrectChoice float
\choice string
\choice Non of the above
\end{checkboxes}
\vspace{1cm}

\newpage
\question What is the output of \code{y\_list} if \code{y\_list = [x**2 for x in range(4)]}?
\begin{checkboxes}
\choice [1, 4, 9, 16]
\choice [2, 4, 6, 8]
\CorrectChoice [0, 1, 4, 9]
\choice [0, 2, 4, 6]
\end{checkboxes}
\vspace{1cm}


\question Which of the following creates a list of length 10 with random numbers between -5 and 5?
\begin{checkboxes}
\choice \code{x = [(np.random.random()-0.5)*2*5 for i in range(10)]}
\choice \code{x = [(np.random.random()-0.5)*5 for i in range(10)]}
\choice \code{x = [(np.random.random()-1)*5 for i in range(10)]}
\choice \code{x = [(np.random.random()-0.5)*2*5 for i in range(9)]}
\end{checkboxes}
\vspace{1cm}

\question If you have two particles at position (1,7) and (2,9), which of the following code examples then gives you the distance between them?

\begin{checkboxes}

\choice 
\begin{lstlisting}
import numpy as np
def distance(a, b, i, j):
    c = a - b
    h = i - j
    d = np.sqrt(c**2 + h**2)

r = distance(1, 2, 7, 9)
print r
\end{lstlisting}

\choice
\begin{lstlisting}
import numpy as np
def distance(a, b, i, j)
    c = a - b
    h = i - j
    d = np.sqrt(c**2 + h**2)
    return d

r = distance(1, 2, 7, 9)
print r
\end{lstlisting}

\CorrectChoice
  \begin{lstlisting}
import numpy as np
def distance(a, b, i, j):
    c = a - b
    h = i - j
    d = np.sqrt(c**2 + h**2)
    return d

r = distance(1, 2, 7, 9)
print r
\end{lstlisting}

\newpage
\choice
\begin{lstlisting}
import numpy as np 
def distance(a, b, i, j):
    c = a - b
    h = i - j
        d = np.sqrt(c**2 + h**2)
    return d

r = distance(1, 7, 2, 9)
print r
\end{lstlisting}

\end{checkboxes}
\vspace{1cm}

\end{questions}





\iffalse
\begin{questions}
\question
\begin{checkboxes}
\choice 
\choice
\choice
\choice
\end{checkboxes}
\end{questions}
\fi




\end{document}