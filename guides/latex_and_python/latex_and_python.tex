\documentclass{article}

\usepackage[utf8]{inputenc}

% Packages
\usepackage{amsmath,amssymb}
\usepackage{bm}% boldmath
\usepackage{listings} % Code block (source code) \begin{lstlisting} 
\usepackage{natbib}
\usepackage{graphicx}
\usepackage{lmodern}
\usepackage[usenames,dvipsnames,svgnames,table]{xcolor}
\usepackage[textwidth=16cm,textheight=23cm]{geometry}

%\usepackage{inconsolata} % New monospace font

% URL
\usepackage{url}
\usepackage[colorlinks=true, a4paper=true, pdfstartview=FitV, linkcolor=blue, citecolor=blue, urlcolor=blue]{hyperref}

% Figures
\usepackage[font=small, labelfont=bf]{caption}
\usepackage{subfig} % Subfigures. Uses \subfloat[captions text]{figure}

% Tables
\usepackage{booktabs}   % Allows the use of \toprule, \midrule and \bottomrule in tables for horizontal lines
\newcommand{\ra}[1]{\renewcommand{\arraystretch}{#1}} % spaces in tables

% Itemize
\usepackage{enumitem}

% Commands
%\newcommand{\code}[1]{\texttt{#1}} % \code{inline code}
\newcommand{\code}[1]{{\small\ttfamily #1}} % \code{inline code}
\newcommand{\expval}[1]{\langle #1 \rangle} %
\renewcommand{\theequation}{\arabic{section}.\arabic{equation}} % Book format equation
\renewcommand{\thefigure}{\arabic{section}.\arabic{figure}} % Book format figure
\renewcommand{\vec}[1]{{\bf #1}}

% Set page attribution
\setlength{\parindent}{0pt}


% PSTRICKS
\usepackage{pstricks,pst-node,pst-tree} % includes graph additions
\usepackage{pst-pdf} % Compiles the pictures
\usepackage{pst-node}
\usepackage{pst-plot}
\usepackage{pst-3dplot}
%\usepackage{pstricks-add,babel}


\definecolor{lbcolor}{rgb}{0.9,0.9,0.9}  

\lstset{
language=Python,                        % Code langugage
commentstyle=\color{gray},              % Comments font
basicstyle=\small\ttfamily,             % Code font, 
keywordstyle=\bfseries\color{blue},
stringstyle=\color{orange},
numbers=left,                           % Line nums position
numberstyle=\tiny,                      % Line-numbers fonts
stepnumber=1,                           % Step between two line-numbers
numbersep=5pt,                          % How far are line-numbers from code
frame=none,                             % A frame around the code
tabsize=4,                              % Default tab size
captionpos=b,                           % Caption-position = bottom
breaklines=true,                        % Automatic line breaking?
breakatwhitespace=false,                % Automatic breaks only at whitespace?
showspaces=false,                       % Dont make spaces visible
showstringspaces=false,                 % Dont make spaces visible in strings
showtabs=false,                         % Dont make tabls visible
belowskip=8pt,
morekeywords={range, xrange, as},
backgroundcolor=\color{lbcolor},  
showstringspaces=false
}


\usepackage{fancyhdr}
\fancyhead[L]{MolStat2016}
\fancyhead[R]{Name (Ku-username)}
\pagestyle{fancy}

\usepackage{fancyvrb}
\usepackage{color}

% ***************************************************
% HEADER INFORMATION

\title{Writing Python-code in latex}
\author{Molecular Statistics}
\date{2016}

% ***************************************************
% ***************************************************
% BEGIN DOCUMENT
% ***************************************************


\begin{document}
\maketitle\thispagestyle{fancy}


For your final exam you will need to write a report as well as the code associated with your study. 
Some of you will probably want to write the report in \LaTeX. 
We have made this little document to show how this can be done so that your report comes out looking nice and clean.

When you have written up some Python code that you want to include in your report you can do this in the \verb+\lstlisting+ environment.

\begin{Verbatim}[commandchars=+\(\)]
\(+color(blue)begin){lstlisting}
(+color(orange)# you can use comments under way to remind yourself later what the code should do.)
(+color(orange)print "I want to print this statement" # you can also comment after a line )
\(+color(blue)end){lstlisting}
\end{Verbatim}

The output of the above \LaTeX \ command is as follows:

\begin{lstlisting}
# you can use comments underway to remind yourself later what the code should do.
print "I want to print this statement" # you can also comment after a line
\end{lstlisting}

This means that you can directly copy-paste the code from you Python program into you final-report.\\

In order to help you set up the proper environment for the Python coding we provide the file \code{preamble.tex}, where the relevant \code{usepackages} have been imported. to import this simply write \verb+\usepackage[utf8]{inputenc}

% Packages
\usepackage{amsmath,amssymb}
\usepackage{bm}% boldmath
\usepackage{listings} % Code block (source code) \begin{lstlisting} 
\usepackage{natbib}
\usepackage{graphicx}
\usepackage{lmodern}
\usepackage[usenames,dvipsnames,svgnames,table]{xcolor}
\usepackage[textwidth=16cm,textheight=23cm]{geometry}

%\usepackage{inconsolata} % New monospace font

% URL
\usepackage{url}
\usepackage[colorlinks=true, a4paper=true, pdfstartview=FitV, linkcolor=blue, citecolor=blue, urlcolor=blue]{hyperref}

% Figures
\usepackage[font=small, labelfont=bf]{caption}
\usepackage{subfig} % Subfigures. Uses \subfloat[captions text]{figure}

% Tables
\usepackage{booktabs}   % Allows the use of \toprule, \midrule and \bottomrule in tables for horizontal lines
\newcommand{\ra}[1]{\renewcommand{\arraystretch}{#1}} % spaces in tables

% Itemize
\usepackage{enumitem}

% Commands
%\newcommand{\code}[1]{\texttt{#1}} % \code{inline code}
\newcommand{\code}[1]{{\small\ttfamily #1}} % \code{inline code}
\newcommand{\expval}[1]{\langle #1 \rangle} %
\renewcommand{\theequation}{\arabic{section}.\arabic{equation}} % Book format equation
\renewcommand{\thefigure}{\arabic{section}.\arabic{figure}} % Book format figure
\renewcommand{\vec}[1]{{\bf #1}}

% Set page attribution
\setlength{\parindent}{0pt}


% PSTRICKS
\usepackage{pstricks,pst-node,pst-tree} % includes graph additions
\usepackage{pst-pdf} % Compiles the pictures
\usepackage{pst-node}
\usepackage{pst-plot}
\usepackage{pst-3dplot}
%\usepackage{pstricks-add,babel}


\definecolor{lbcolor}{rgb}{0.9,0.9,0.9}  

\lstset{
language=Python,                        % Code langugage
commentstyle=\color{gray},              % Comments font
basicstyle=\small\ttfamily,             % Code font, 
keywordstyle=\bfseries\color{blue},
stringstyle=\color{orange},
numbers=left,                           % Line nums position
numberstyle=\tiny,                      % Line-numbers fonts
stepnumber=1,                           % Step between two line-numbers
numbersep=5pt,                          % How far are line-numbers from code
frame=none,                             % A frame around the code
tabsize=4,                              % Default tab size
captionpos=b,                           % Caption-position = bottom
breaklines=true,                        % Automatic line breaking?
breakatwhitespace=false,                % Automatic breaks only at whitespace?
showspaces=false,                       % Dont make spaces visible
showstringspaces=false,                 % Dont make spaces visible in strings
showtabs=false,                         % Dont make tabls visible
belowskip=8pt,
morekeywords={range, xrange, as},
backgroundcolor=\color{lbcolor},  
showstringspaces=false
}

+ before \verb+\begin{document}+\\

\begin{center}\large{The project report should include}\end{center}

The report is done individually, should be written in either Danish or English and be ca. 10
and no more than 20 pages long. The code associated with the study should be uploaded on the course
website or emailed to the instructor.
The report should be written with enough theoretical and technical background so any of your colleagues
can understand what you have done, and be able to reproduce your results.


\begin{abstract}
An abstract is a brief \textbf{summary} of a research article, thesis,
review, conference proceeding or any in-depth analysis of a particular subject or discipline, and is
often used to help the reader quickly ascertain the paper’s purpose.
\end{abstract}

\section{Introduction and theory}
 A theoretical introduction to the subject along with motivation for
the work. This includes a description of the theory and algorithms, as well as figures that could
help with the understanding of the subject.

\section{Implementation}
 Comment on what the code does and how you implemented the code. You are welcome to use code examples.
 \section{Results and discussion}
 Presentation of your results (use figures, pictures and tables). Discussion
of the results. Do the results behave as expect, are there outliers and do you trust your results?
\section{Conclusion} Summary of the most important findings. Perspective of your results.
\begin{itemize}
\item Did you achieve the goal described in the introduction?
\item What would be the next step?
\item Could one improve on your calculations?
\end{itemize}

\vspace{40pt}
You can use Bib\TeX \ for creating the references. You should cite all books \cite{thinkPyhton} and web pages \cite{The_Listings_Package} that you've used in order to complete the report and code. 
Example of how to write the \code{.bib} file will be on the web page. 

\bibliography{ref.bib} %you need to download ref.bib in order to compile the referencesin this example
\bibliographystyle{agsm}





% ***************************************************
% END DOCUMENT
% ***************************************************

\end{document}
