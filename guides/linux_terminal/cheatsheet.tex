\documentclass{article}

\usepackage[utf8]{inputenc}

% Packages
\usepackage{amsmath,amssymb}
\usepackage{bm}% boldmath
\usepackage{listings} % Code block (source code) \begin{lstlisting} 
\usepackage{natbib}
\usepackage{graphicx}
\usepackage{lmodern}
\usepackage[usenames,dvipsnames,svgnames,table]{xcolor}
\usepackage[textwidth=16cm,textheight=23cm]{geometry}

%\usepackage{inconsolata} % New monospace font

% URL
\usepackage{url}
\usepackage[colorlinks=true, a4paper=true, pdfstartview=FitV, linkcolor=blue, citecolor=blue, urlcolor=blue]{hyperref}

% Figures
\usepackage[font=small, labelfont=bf]{caption}
\usepackage{subfig} % Subfigures. Uses \subfloat[captions text]{figure}

% Tables
\usepackage{booktabs}   % Allows the use of \toprule, \midrule and \bottomrule in tables for horizontal lines
\newcommand{\ra}[1]{\renewcommand{\arraystretch}{#1}} % spaces in tables

% Itemize
\usepackage{enumitem}

% Commands
%\newcommand{\code}[1]{\texttt{#1}} % \code{inline code}
\newcommand{\code}[1]{{\small\ttfamily #1}} % \code{inline code}
\newcommand{\expval}[1]{\langle #1 \rangle} %
\renewcommand{\theequation}{\arabic{section}.\arabic{equation}} % Book format equation
\renewcommand{\thefigure}{\arabic{section}.\arabic{figure}} % Book format figure
\renewcommand{\vec}[1]{{\bf #1}} % Lars likes this better than arrow

% Set page attribution
\setlength{\parindent}{0pt}


% PSTRICKS
\usepackage{pstricks,pst-node,pst-tree} % includes graph additions
\usepackage{pst-pdf} % Compiles the pictures
\usepackage{pst-node}
\usepackage{pst-plot}
\usepackage{pst-3dplot}
%\usepackage{pstricks-add,babel}




\lstset{
language=Python,                        % Code langugage
commentstyle=\color{gray},              % Comments font
basicstyle=\small\ttfamily,             % Code font, Examples: \footnotesize, \ttfamily
keywordstyle=\bfseries\color{blue},
stringstyle=\color{orange},
numbers=left,                           % Line nums position
numberstyle=\tiny,                      % Line-numbers fonts
stepnumber=1,                           % Step between two line-numbers
numbersep=5pt,                          % How far are line-numbers from code
frame=none,                             % A frame around the code
tabsize=4,                              % Default tab size
captionpos=b,                           % Caption-position = bottom
breaklines=true,                        % Automatic line breaking?
breakatwhitespace=false,                % Automatic breaks only at whitespace?
showspaces=false,                       % Dont make spaces visible
showstringspaces=false,                 % Dont make spaces visible in strings
showtabs=false,                         % Dont make tabls visible
belowskip=8pt,
morekeywords={range, xrange},
% backgroundcolor=\color{yellow}
% emph={[2]root,base}
% morekeywords={one,two,three,four,five,six,seven,eight,
}


%commentstyle=\color{gray},              % Comments font
%basicstyle=\small,                      % Code font, Examples: \footnotesize, \ttfamily



%basicstyle=\footnotesize\ttfamily,
%keywordstyle=\bfseries\color{green!40!black},
%commentstyle=\itshape\color{purple!40!black},
%identifierstyle=\color{blue},
%stringstyle=\color{orange},






\usepackage{multicol} % \begin{columns}
\usepackage{array}
\usepackage{hyperref}
% ***************************************************
% HEADER INFORMATION

\title{Unix cheat-sheet}
\author{Molecular Statistics}
\date{Last edit: March 2016}

% ***************************************************

\begin{document}
\newcommand{\ti}[1]{\texttt{\textit{#1}}}
\renewcommand{\arraystretch}{2}

% ***************************************************
% BEGIN DOCUMENT
% ***************************************************

\maketitle

\section{Introduction}
Using the terminal to navigate through directories and file trees as well as managing files might seem excessive at first, but soon you will find that managing your data will become must faster this way. In order for this to be true, we here provide some of the most commonly used commands. We encourage all to go online at search "unix commands" if you are interested in becoming even more efficient. 

\section{Common Commands}
\subsection{Help on any Unix command}
\begin{table}[h]
\begin{tabular}{ll}
 	\texttt{man {command}} &	Type man ls to read the manual for the ls command. \\
	\texttt{man {command} > {filename}} &	Redirect help to a file to download. \\
	\texttt{whatis {command}} &	Give short description of command.\\
\end{tabular}
\end{table}

\subsection{File Commands}
\begin{table}[h]
\begin{tabular}{>{\ttfamily}l l}
	ls & list directory \\
	ls -al & list with hidden files \\
	cd \textit{dir} & change directory to \texttt{\textit{dir}}\\
	cd & change to home \\
	cd - & change to previous directory \\
	pwd & show current directory \\
	rm \textit{file} & remove \texttt{\textit{file}} \\
	rm -r \textit{dir} & remove directory \\
	cp \textit{file1} \textit{file2} & copy \ti{file1} to \ti{file2} (\ti{file2} is overwritten) \\
	cp -r \ti{dir1} \ti{dir2} & copy \ti{dir1} to \ti{dir2}; create \ti{dir2} if it doesnt exist \\
	cp \ti{file1} \ti{dir1} & copy \ti{file1} to \ti{dir1} \\
	mv \ti{file1} \ti{dir1} & move \ti{file1} to \ti{dir1} \\
	mv \ti{file1} \ti{file2} & rename \ti{file1} to \ti{file2} \\
	touch \ti{file} & create or update \ti{file} \\
	cat  > \ti{file} & place output in \ti{file} \\
	more \ti{file} & output: contents of \ti{file} \\
	head \ti{file} & first 10 lines of \ti{file} \\
	tail \ti{file} & last 10 lines of \ti{file} \\
	tail -f \ti{file} & follow end of \ti{file} as it grows \\
\end{tabular}
\end{table}

\subsection{Permissions and searching}
\begin{tabular}{>{\ttfamily}ll}
\texttt{chmod} \ti{octal} \ti{file} & change permission of \ti{file} to \ti{octal} \\ 
4 & read (r) \\ 
2 & write (w) \\ 
3 & execute (x) \\ 
Examples &  \\ 
\texttt{chmod 777} & rwx for all \\ 
\texttt{chmod 755} & rwx for owner, rx for group and world \\ 
\textnormal{For more options see} \texttt{chmod man} &  \\ 
grep \ti{pattern} \ti{files} & search for \ti{pattern} in \ti{files} \\ 
comman | grep \ti{pattern} & search for \ti{pattern} in the outpu of \texttt{command} \\ 
locate \ti{file} & find all instances of \ti{file} \\
\end{tabular} 

\subsection{Shortcuts}
\begin{tabular}{ll}
Ctrl+C & Stop curent command \\ 
Ctrl+Z & Pause current command. Restart with \texttt{fg} to run in forground and \texttt{bg} in background \\ 
Ctrl+D & log out - simlar to \texttt{exit} \\ 
Ctrl+U & Erase line \\ 
Ctrl+R & Type to bring up recent command \\ 
\end{tabular} 
% ***************************************************
% END DOCUMENT
% ***************************************************

\subsection{Regular expressions}
An expression is a string of characters.  A Regular Expression (RE) are sets of characters that match or specify a pattern. A RE contains one or more of the following

\begin{itemize}
\item \textit{A character set}. These are the characters retaining their literal meaning. The simplest type of Regular Expression consists only of a character set, with no metacharacters.
\item \textit{An anchor}. These designate (anchor) the position in the line of text that the RE is to match. For example \$ is a anchor.
\item \textit{Modifiers}. These expand or narrow (modify) the range of text the RE is to match. Modifiers include the asterisk, brackets, and the backslash.
\end{itemize}

\begin{tabular}{ll}
* & matches anything preceeding it one ore more times \\ 
Example & "1133*" matches 11 + one or more 3's \\ 
. & matches any character except new line \\ 
Example & "13." matches 13+ at least one of any character \\ 
$\wedge$ & matches beginning of line \\ 
\$ & matches end of line \\ 
$[...]$ & encloses set of characters to match \\ 
Example & [xyz] matches any one of the characters x,y or z \\ 
$\backslash$ & escapes a special character - it is interpreted literally \\ 
 $\backslash < ...\backslash >$ & marks word boundaries \\ 			
Example &  $\backslash < \text{the} \backslash >$ matches "the" but not "them", "there", "other" etc. \\ 
\end{tabular} 

\vspace{50pt}



References:\\
\url{https://www.rain.org/~mkummel/unix.html} \\
\url{https://rumorscity.com/wp-content/uploads/2014/08/10-Linux-Unix-Command-Cheat-Sheet-021.jpg}\\
\url{http://tldp.org/LDP/abs/html/x17129.html}\\
\end{document}